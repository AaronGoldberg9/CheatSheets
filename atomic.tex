\documentclass[10.5pt,letterpaper]{article}
\usepackage{amsfonts}
\usepackage{amsmath}
\usepackage{amsfonts}
\usepackage{amssymb}
\usepackage{bbm}
\usepackage{fullpage}

\title{Useful Identities in Atomic Physics}
\author{Nicol\'as Quesada\\{\small \sf Instituto de F\'isica, Universidad de Antioquia}}
\date{}

\begin{document}
\maketitle
\thispagestyle{empty}
%\subsection*{Notation and Conventions}
\noindent The quantum numbers of the energy eigenstates of the non-relativistic hydrogenoid atom are: $n$ (energy), $l$ (angular momentum) and $m$ ($z$ projection of angular momentum). $\langle \cdot \rangle$ means average over \emph{energy} eigenstates.

\subsection*{Scales Constants and Special Values of Hydrogenoid Wave Functions}

Energies for the Coulomb Potential ($V(\textbf{r})=-\frac{1}{4 \pi \epsilon_0}\frac{Z e^2}{r})$ are $E_n=-\frac{\mu}{2n^2}\left( \frac{Ze^2}{4 \pi \epsilon_0 \hbar} \right)^2=-\frac{e^2}{4 \pi \epsilon_0 a_0} \frac{Z^2}{2 n^2}=-\frac{1}{2} \mu c^2 \frac{(Z \alpha)^2}{n ^2}$.\\

\noindent Fine structure constant: $\alpha=\frac{e^2}{4 \pi \epsilon_0 \hbar c}$. Bohr radius: $a_0=4 \pi \epsilon_0 \hbar^2/(\mu e^2)$. \space \space \space
$|\psi_{n l m}(0)|^2=\frac{Z^3}{\pi a_0^3 n^3} \delta_{l}^0 \delta_m^0$.

\subsection*{Expected Values, the Virial Theorem and the Gamma Function}

Virial Theorem (valid for \emph{any} potential): If $H=T(\textbf{p})+V(\textbf{r})$ and $T(\textbf{p})=\frac{\textbf{p}^2}{2\mu}$ then $2\langle T \rangle=\langle \textbf{r} \cdot \nabla V \rangle$ .\\

\noindent Expectation values for the Coulomb potential: $\left\langle \frac{1}{r} \right\rangle=\frac{Z}{a_0 n^2}$ , $\left\langle \frac{1}{r^2} \right\rangle=\frac{Z^2}{a_0^2 n^3 (l+1/2)}$ .\\

\noindent Recursion Relation: $0=\frac{s}{4}\left[(2 l+1)^2-s^2 \right] \left( \frac{a_0}{Z} \right)^2 \langle r^{s-2} \rangle-(2s+1)\left(\frac{a_0}{Z} \right) \langle r^{s-1} \rangle+\frac{s+1}{n^2} \langle r^s \rangle$ .\\

\noindent $\Gamma(z) = \int_0^\infty  t^{z-1} e^{-t}\,dt $ \space , \space $\Gamma(n+1)=n!$\space , \space $ \Gamma(1-z) \; \Gamma(z) = {\pi \over \sin{(\pi z)}} \,\! $ \space , \space  $\Gamma(z) \; \Gamma\left(z + \frac{1}{2}\right) = 2^{1-2z} \; \sqrt{\pi} \; \Gamma(2z). \,\! $


\subsection*{Spherical Harmonics, Wigner 3$j$ Symbols and Clebsch -- Gordan Coefficients}
$\int_0^{2 \pi} d\phi \int_0^{\pi} \sin \theta d\theta \ \mathcal{Y}_{l_1}^{m_1}(\theta,\phi) \mathcal{Y}_{l_2}^{m_2}(\theta,\phi) \mathcal{Y}_{l_3}^{m_3}(\theta,\phi)=\sqrt{\frac{(2l_1+1)(2l_2+1)(2l_3+1)}{4 \pi}} \begin{pmatrix}
  l_1 & l_2 & l_3\\
  0 & 0 & 0
\end{pmatrix}
\begin{pmatrix}
  l_1 & l_2 & l_3\\
  m_1 & m_2 & m_3
\end{pmatrix}
$.
\\

\noindent Wigner 3$j$ --- Clebsch--Gordan (CG) relation:
$
\begin{pmatrix}
  j_1 & j_2 & j_3\\
  m_1 & m_2 & m_3
\end{pmatrix}
\equiv \frac{(-1)^{j_1-j_2-m_3}}{\sqrt{2j_3+1}} \langle j_1 m_1 j_2 m_2 | j_3 \, {-m_3} \rangle.$\\

\noindent Selection rules for Wigner 3$j$ Symbol $\begin{pmatrix}  l_1 & l_2 & L\\   m_1 & m_2 & -M \end{pmatrix}$ (they are identical to CG Selection Rules):

\noindent 
\begin{tabular}{llll}
$-l_i \leq m_i \leq l_i, $ & $m_1+m_2=M$, & $|l_1-l_2|\leqslant L \leqslant l_1+l_2$ & $l_1+l_2+L \in \mathbbm{Z}$  \.\ 
\end{tabular}
\\

\noindent Spherical components of a cartesian vector $\vec e=(e_x,e_y,e_z)$: $e_{\pm 1}=\mp \frac{1}{\sqrt{2}} \left( e_x\pm i e_y \right)$ and $e_0=e_z$.\\

\noindent Ladder Operators: $ \hat L_{\pm}\equiv \hat L_x\pm i \hat L_y$. \space \space \space $\hat L_{\pm} |l,m \rangle=\hbar \sqrt{l(l+1)-m(m\pm1)}|l,m_l\pm1 \rangle$.

\subsection*{References}

%\begin{itemize}

B.H. Bransden, C.J. Joachain \textit{Physics of Atoms and Molecules}\\
G.B. Arfken, H.J. Weber \textit{Mathematical Methods for Physicists}\\
E. W. Weisstein, ``Wigner 3$j$-Symbol.'' \verb|http://mathworld.wolfram.com/Wigner3j-Symbol.html|\\
S. Jeon, \textit{Lecture Notes for Quantum Physics} II. \verb|http://www.physics.mcgill.ca/~jeon/Phys457/|
%\end{itemize}

\end{document}

