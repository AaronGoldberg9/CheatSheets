\documentclass[11pt,letterpaper]{article}
\usepackage{graphicx}
\usepackage[fleqn]{amsmath}
\usepackage{amssymb}
\usepackage{amsmath}
\usepackage{fullpage}
\usepackage{comment}
%\usepackage{setspace}

%\doublespacing

\newcommand{\ket}[1]{| #1 \rangle}
\newcommand{\bra}[1]{\langle #1 |}
\newcommand{\braket}[1]{\langle #1 \rangle}
\newcommand{\hc}{\text{H.c.}}
\newcommand{\cc}{\text{c.c.}}
\newcommand{\vac}{\text{vac}}
\newcommand{\h}{\hat}
\newcommand{\ii}{\mathrm{i}}
\newcommand{\thf}{\text{TopHat}}
\newcommand{\eq}[1]{\begin{align}#1\end{align}}
\DeclareMathOperator{\tr}{tr}



\title{Fourier Transforms and Discrete Fourier Transforms}
\author{Nicol\'as Quesada}

\begin{document}
\maketitle
The Fourier Transform (FT) of a function can be defined as
\eq{
F(\omega)=\int_{-\infty}^\infty f(t) e^{-\ii \omega t}.
}
We can approximate this integral by a Riemann sum
\eq{
F(\omega) \approx \int_{-\infty}^\infty dt \left(\sum_{j=0}^{N-1} f(t_j) \delta(t-t_j) \Delta t \right) e^{-\ii \omega t}=\Delta t \sum_{j=0}^{N-1} f(t_j) e^{-\ii \omega t_j}.
}
where we use the following definitions
\eq{\label{defs}
\Delta t=\frac{t_f-t_i}{N} \text{ and } t_j=t_i+j \Delta t.
}
Note however that $t_0=t_i$ but $t_N=t_i/N+(1-1/N) t_f \neq t_f$. Nevertheless, when $N \gg 1$ one has $t_N \to t_f$. Note also that if $N$ is even then $t_{N/2}=(t_f+t_i)/2$. 

Now let us pick $\omega$ to be a multiple of some (at the moment) undetermined frequency $\Delta \omega$, $\omega=k \Delta \omega$ and write $t_j=t_i+j \Delta t$
\eq{
F(k \Delta \omega)&\approx \Delta t\sum_{j=0}^{N-1} f(t_j) e^{-\ii k \Delta \omega (t_i+j \Delta t)}=\Delta t e^{-\ii k \Delta \omega t_i}\sum_{j=0}^{N-1} f(t_j) e^{-\ii k j \Delta \omega  \Delta t}.
}
It is convenient to pick $\Delta \omega = 2 \pi/(\Delta t N)=2 \pi /(t_f-t_i)$ to get
\eq{\label{almost}
F(k \Delta \omega)&\approx \Delta t e^{-\ii k \Delta \omega t_i}\sum_{j=0}^{N-1} f(t_j) e^{-2 \pi \ii k  j/N}.
}
If we define $x_j=f(t_j)$ then we can identify
\eq{\label{DFT}
X_k=\sum_{j=0}^{N-1} x_j e^{-2 \pi \ii k  j/N}\equiv \mathcal{F}\left(\{x_j \}\right),
}
as the Discrete Fourier transform of the sequence $x_j$ and write the FT  $F(k \Delta \omega)$ in terms of the DFT
\eq{\label{quasi}
F(k \Delta \omega)&\approx \Delta t e^{-\ii k \Delta \omega t_i} X_k.
}
Typically one will pick $t_i=-t_f$ in which case the prefactor in Eq. (\ref{almost}) becomes
\eq{
e^{-\ii k \Delta \omega t_i}=e^{-\ii k \frac{(2 \pi)}{(-t_i-t_i)} t_i}=e^{\ii k \pi}=(-1)^k.
}

Now let us assume that the function $f(t)$ is real and symmetric, $f(t)=f(-t)$. Then, one can easily show that the function $F(\omega)$ is real. Does the same hold true for the sequences $x_j$ and $X_k$? Indeed, it does, if one defines a symmetric sequence to satisfy $x_j=x_{N-j}$ then one can, using the definition of DFT in Eq. (\ref{DFT}), show that $X_k=X_k^*$. Now, how one should sample $f(t)$ in such away that the Fourier transform obtained by using the DFT satisfies the type of symmetries mentioned before?
It turns out that by sampling as in Eq. (\ref{defs}) with $t_f=-t_i$ one gets the desired property since
\eq{
x_{N-j}&=f(t_i+(N-j)\Delta t)=f(t_i+N\Delta t-j\Delta t)=f(t_i+t_f-t_i-j \Delta t  )\\
&=f(t_f-j\Delta t )=f(-t_i-j \Delta t )=f(t_i+j \Delta t )=x_j. \nonumber
}
In the last two equalities we used the fact that $t_i=-t_f$ and that $f(t)=f(-t)$.

One final question is how to get rid of the annoying factor $(-1)^k$ in Eq. (\ref{quasi}). We can rewrite it as
\eq{
F(k \Delta \omega)&\approx \Delta t e^{-\ii k \Delta \omega t_i} X_k=\Delta t \sum_{j=0}^{N-1} f(t_j) e^{-2 \pi \ii k  j/N} e^{-\ii \pi k }=\Delta t \sum_{j=0}^{N-1} f(t_j) e^{-2 \pi \ii k  (j+N/2)/N}.
}
Now note that the $j$ indices are only defined modulo $N$, if we let $j \to j+N$ we get the same DFT since $e^{2 \pi \ii k(j+N)/N}=e^{2 \pi \ii k j /N}$, thus we can identify
\eq{
x_{(j+N/2) \text{mod} N} = f(t_j),
} 
and write
\eq{
F(k \Delta \omega)&\approx \Delta t \mathcal{F}\left(\{x_{(j+N/2) \text{mod} N} \}\right),
}
which gives directly the FT in terms of the DFT of the (re-arranged) sampled values.
\end{document}
